\documentclass[hidelinks, 12pt]{article}
\usepackage[italian]{babel}
\usepackage{natbib}
\usepackage{url}
\usepackage[utf8x]{inputenc}
\usepackage{amsmath}
\usepackage{graphicx}
\graphicspath{{images/}}
\usepackage{parskip}
\usepackage{fancyhdr}
\usepackage{vmargin}
\usepackage{float}
\usepackage{hyperref}
\usepackage{subfig}
\setmarginsrb{3 cm}{2.5 cm}{3 cm}{2.5 cm}{1 cm}{1.5 cm}{1 cm}{1.5 cm}

\title{Amazon Reviews \\ Sentiment Analysis}				% Title
\author{Basso Matteo \\ Ferri Marco}								% Author
\date{Luglio 2019}				% Date

\makeatletter
\let\thetitle\@title
\let\theauthor\@author
\let\thedate\@date
\makeatother

\pagestyle{fancy}
\fancyhf{}
\rhead{Basso M., Ferri M.}
\lhead{\thetitle}
\cfoot{\thepage}

\begin{document}
	
%%%%%%%%%%%%%%%%%%%%%%%%%%%%%%%%%%%%%%%%%%%%%%%%%%%%%%%%%%%%%%%%%%%%%%%%%%%%%%%%%%%%%%%%%

\begin{titlepage}
	\centering
	\vspace*{0.5 cm}
	\includegraphics[scale = 0.75]{images/LogoBicocca.pdf}\\[1.0 cm]	% University Logo
	\textsc{\LARGE Università degli studi di}\\[0.2 cm]
	\textsc{\LARGE Milano-Bicocca}\\[2.0 cm]	% University Name
	\textsc{\Large F1801Q127}\\[0.5 cm]				% Course Code
	\textsc{\large Data Analytics}				% Course Name
	\rule{\linewidth}{0.2 mm} \\[0.4 cm]
	{ \huge \bfseries \thetitle}\\
	\rule{\linewidth}{0.2 mm} \\[1.5 cm]
	
	\begin{minipage}{0.4\textwidth}
		\begin{flushleft} \large
			\emph{Studenti:}\\
			\theauthor
		\end{flushleft}
	\end{minipage}~
	\begin{minipage}{0.4\textwidth}
		\begin{flushright} \large
			\emph{Matricole:} \\
			807628 \\ 807130
		\end{flushright}
	\end{minipage}\\[2 cm]
	
	{\large \thedate}\\[2 cm]
	
	\vfill
	
\end{titlepage}

%%%%%%%%%%%%%%%%%%%%%%%%%%%%%%%%%%%%%%%%%%%%%%%%%%%%%%%%%%%%%%%%%%%%%%%%%%%%%%%%%%%%%%%%%


\null\vspace{\stretch{1}}
\section*{\centering Abstract}

Lorem ipsum dolor sit amet.

\vspace{\stretch{2}} \null

\clearpage


%%%%%%%%%%%%%%%%%%%%%%%%%%%%%%%%%%%%%%%%%%%%%%%%%%%%%%%%%%%%%%%%%%%%%%%%%%%%%%%%%%%%%%%%%

\tableofcontents
\clearpage
\listoffigures
\listoftables
\pagebreak

%%%%%%%%%%%%%%%%%%%%%%%%%%%%%%%%%%%%%%%%%%%%%%%%%%%%%%%%%%%%%%%%%%%%%%%%%%%%%%%%%%%%%%%%%



\section{Introduzione}

\subsection{Dataset}

\subsection{Obiettivi}

\subsection{Ipotesi e assunzioni}

\subsection{Software}



\clearpage



\section{Basic Analysis}

\subsection{Schema}

\subsection{Dimensioni}

\subsection{Distribuzioni}

\subsection{Analisi business-oriented}



\clearpage



\section{Network Analysis}

\subsection{Struttura della rete}

\subsection{Grado dei nodi}

\subsection{Misure di centralità ?}



\clearpage



\section{Sentiment Analysis}


\subsection{Assunzioni}
\label{sec:bn-assumptions}

\subsection{Binarizzazione}

\subsection{Undersampling}

\subsection{Elaborazione del testo}

\subsection{Parole più usate}



\clearpage



\section{Sentiment Prediction}

\subsection{Pesatura dei termini (TF-IDF)}

\subsection{Termini più rilevanti}

\subsection{Modelli di predizione}

\subsubsection{Random Forest}
\subsubsection{Naive Bayes}
\subsubsection{SVM}

\subsection{Pipeline}



\clearpage



\section{Aspect Based Sentiment Analysis}

\subsection{Elaborazione del testo}

\subsection{Estrazione degli aspetti}

\subsection{Identificazione del sentiment}

\subsection{Risultati}



\clearpage



\section{Collaborative Filtering}

\subsection{Funzionamento}

\subsection{Risultati}



\clearpage



\section{Web Demo}
\label{sec:ui}


\subsection{Architettura}

L'interfaccia web è stata sviluppata utilizzando l'architettura a 3 layer, con separazione di frontend, backend e database.

Il database utilizzato in fase di lettura è quello fornito inizialmente, senza alcuna modifica. Esso consiste quindi in un file SQLite interrogabile e modificabile semplicemente tramite un web server. Questo risulta particolarmente utile per fornire i dettagli dei giocatori ed eventualmente dei team così che l'utente possa visualizzarli e sceglierli attraverso l'opportuna interfaccia.

Per lo sviluppo del backend è stato deciso di utilizzare l'engine Javascript tramite il popolare progetto Node.js \cite{site:nodejs}. Esso è in grado di agire come middleware tra il frontend e il database, separando al meglio le logiche di manipolazione del dato. \'E inoltre incaricato di chiamare adeguatamente lo script R per la predizione del vincitore della partita e per svolgere le inferenze richieste.

Il frontend è invece sviluppato utilizzando la libreria Javascript React.js \cite{site:react}



\subsection{Sentiment Prediction}

\subsection{Aspect Based Sentiment Analysis}



\clearpage



\section{Conclusioni}





\newpage
\bibliographystyle{plainurl}
\bibliography{biblist}
	
\end{document}
